
% Alle Konfigurationsbefehle sind optional. Fehlende Befehle fueheren einfach
% zu "blank forms".

% Typ der Arbeit/Einstellung. Gueltige Argumente sind:
% bachelor,master,diplom,idp,gr,hiwi,other
% Falls 'other' gewaehlt wird, kann als optionales Argument eine spezielle Art
% von Abschlussarbeit angegeben werden, z.B. \type[Sklave]{other}. Andernfalls
% wird 'Other' als Standardbeschreibung gesetzt.
\type{bachelor}

% Informationen ueber den Studenten. Sollte selbsterklaerend sein.
\anrede{Herr}
\nachname{Okelmann}
\vorname{Peter}
\matrikel{03671011}
\sunhalle{okelmann}
\semester{7}{WiSe\,2018}
\studientelefon{}{0152 0912 8569}
\heimattelefon{}{--}
\studienadresse{Erlenweg 4a}{85386 Eching}
\heimatadresse[adresszusatz=,appartment=]{}{}
\mail{okelmann@in.tum.de}

% Informationen ueber die Arbeit. Sollte selbsterklaerend sein.
\themensteller{\NEThead}
\beginn{12}{2018}
\endt{04}{2019}
\betreuer{Paul Emmerich, Dominik Scholz}
\title{Performance Analysis of VPP}{Performance-Analyse von VPP}
\studiengang{Informatik}


% Falls \type{hiwi} gesetzt wurde, wird die Taetigkeit auf dem Aufnahmeformular
% des Lehrstuhls angegeben.
\taetigkeit{test}
