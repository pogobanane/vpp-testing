\documentclass[NET,a4,12pt,ngerman]{netforms}

\usepackage[utf8]{inputenc}
\usepackage{tumlang}
\usepackage{tumcontact}
\usepackage{scrpage2}

\geometry{%
	top=10mm,
	bottom=10mm,
	left=25mm,
	right=25mm,
	headsep=1.5cm,
	includehead,
}

% Alle Konfigurationsbefehle sind optional. Fehlende Befehle fueheren einfach
% zu "blank forms".

% Typ der Arbeit/Einstellung. Gueltige Argumente sind:
% bachelor,master,diplom,idp,gr,hiwi,other
% Falls 'other' gewaehlt wird, kann als optionales Argument eine spezielle Art
% von Abschlussarbeit angegeben werden, z.B. \type[Sklave]{other}. Andernfalls
% wird 'Other' als Standardbeschreibung gesetzt.
\type{bachelor}

% Informationen ueber den Studenten. Sollte selbsterklaerend sein.
\anrede{Herr}
\nachname{Okelmann}
\vorname{Peter}
\matrikel{03671011}
\sunhalle{sunaccount}
\semester{1}{WiSe\,2018}
\studientelefon{}{tel}
\heimattelefon{}{--}
\studienadresse{strasse}{plz stadt}
\heimatadresse[adresszusatz=,appartment=]{}{}
\mail{okelmann@in.tum.de}

% Informationen ueber die Arbeit. Sollte selbsterklaerend sein.
\themensteller{\NEThead}
\beginn{12}{2018}
\endt{04}{2019}
\betreuer{Paul Emmerich, Dominik Scholz}
\title{Performance Analysis of VPP}{Performance-Analyse von VPP}
\studiengang{Informatik}


% Falls \type{hiwi} gesetzt wurde, wird die Taetigkeit auf dem Aufnahmeformular
% des Lehrstuhls angegeben.
\taetigkeit{test}


\pagestyle{scrheadings}
\clearscrheadfoot
\chead{\TUMheader{1cm}}

\renewcommand{\maketitle}{%
	\begin{center}
		\textbf{\introductoryheadline}%

		\Large%
		\textbf{\thetitle}%
	\end{center}

	\footnotesize%
	\hrule
	\vskip1ex
	\begin{tabular}{ll}
		\thenamelabel: & \thevorname{} \textbf{\thenachname}\\
		\theadvisorlabel: & \hspace*{-.5ex}\thebetreuer\\
		\thesupervisorlabel: & \chairhead\\
		\thebeginlabel: & \thebeginnmonat/\thebeginnjahr\\
		\theendlabel: & \theendmonat/\theendjahr\\
	\end{tabular}
	\vskip1ex
	\hrule
	\vskip4ex
}

\linespread{1.2}
\setlength{\parskip}{.5\baselineskip}

\begin{document}
\maketitle

\subsection*{Topic}

VPP (Vector Packet Processing) \cite{vppwiki:1} is an open source software for
providing efficient network switching and routing.
In contrary to common hardware switches, the whole packet processing is done in
software. This allows a more versatile use of the same router in different
applications and provides flexibility regarding hardware it can run on.

Typically hardware routers are expected to be faster than software routers, but
VPP has several approaches to perform better compared to similar software
projects like Open vSwitch \cite{openvswitch:1} or the Linux Router: As the name indicates, it
processes packages in vectors, in other words, multiple packages at a time which
reduces overhead per package. Furthermore it doesn't use the slow linux network drivers for it's interfaces, but an own one.

The scope of this Bachelor's thesis shall be to measure the performance of VPP
and to evaluate which scenarios are important to be tested.


\subsection*{Approach}

For this Bachelor's Thesis  automated tests shall be implemented to run test
scenarios on the Baltikum testbed. Moongen \cite{moongen:1} shall be used on the
tester to generate loads. The testing shall be reproducable by beeing automated
and shall be comparable by documenting circumstances well.

VPP shall be tested torwards IPv4 versus IPv6, packet sizes, test traffic
patterns, cpu scaling (optionally multi socket NUMA architectures) and size of
the Forwarding and Routing Information Base (BGP size \textgreater 600.000
entries in 2017 \cite{bgphelp:1}). Optionally one advanced scenario can be
analyzed, too, like tunneling, firewall, NAT or virtualization.

Measurement results shall contain package count (per time), package loss,
latency and whitebox testing results using perf: Cpu load and cache misses.

In the end a performance model shall be created (as in \cite{compare-highperf})
and a conclusion shall be made, giving VPP a rough rank relative to real world
routing solutions.


\subsection*{Timetable}

\begin{tabular}{|l|p{10cm}|}
\hline
Deadline after $n$ Months & Task \\ \hline
0.5 & Automate a basic testbench setup and run a test using moongen for
the tester and VPP for the DUT \\ \hline
1.0 & Design and run benchmarking scenarios for basic VPP configurations \\ \hline
1.5 & Create result visualization pipelines and analyze them \\ \hline
2.5 & Improve tests and implement advanced test scenarios \\ \hline
3.0 & Analyze new results and review testing methodology \\ \hline
4.0 & Assemble Thesis from analysiss and testing results \\ \hline
\end{tabular}

\bibliographystyle{IEEEtran}
\bibliography{IEEEabrv,lit}


\end{document}
