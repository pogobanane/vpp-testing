\chapter{Conclusion}

FastClick and \Ac{vpp} are both fast software routers which vectorize
packets during processing and are close to each other performance
wise. MoonRoute on the other hand has a significant performance
advantage over both of them, being 20-30\% faster at routing.

While the performance of \Ac{vpp} v18.10 starts dropping at around 20
\Ac{fib} entries, \Ac{vpp} v16.09 can hold it's highest performance
with up to 200 \Ac{ip4} \Ac{fib} entries. 

Reaching the maximum number of entries \Ac{vpp}'s throughput nearly
halves with \Ac{vpp} v18.10 only supporting up to around 255,000
\Ac{ip4} \Ac{fib} entries. For this number of router table entries it
is remarkably slower than FastClick with $2^{20}$ entries, even though
it is slightly faster with little table entries.

The best advantage of \Ac{vpp} over it's competitors is it's feature
richness. It's during runtime configurable packet processing graph
offers for example different tunneling protocols. Settings allow to
move the main thread to a dedicated CPU core which in turn allows live
inserts of 255k routing table entries with a throughput impact of less
than a percent.

In combination with the rich options to connect \Ac{vpp} to virtual
machines, containers or local high performance applications, it is
well suited for building virtual networks for highly virtualized
environments or implementation of \Ac{vnf}.

Next research steps could include more specific benchmarks regarding
specific protocol features like behavior on receiving control packets.
Furthermore the performance change over different \Ac{vpp} versions
can be analyzed closer by testing a version between 16.09 and 18.10
and the latest v19.01 which was just released during the creation of
this paper. Especially the code changes leading to the performance
differences are of interest.
