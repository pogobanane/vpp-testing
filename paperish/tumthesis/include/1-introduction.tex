\chapter{Introduction}

% microservices -> virtualization -> software routing
% problem: software router slower than specialized hardware

A world of microservices counting on virtualization is more and
more interested in beeing able to move dedicated hardware into
virtualized spaces. But especially high speed networking is
traditionally bound to specialized hardware, because many software
routers can't keep up performancewise.

% solution: faster software routers like VPP
% vpp: cisco

There are software routers though which focus on performance and are
improved in many aspects. Especially frameworks like DPDK help in the
making of fast software routers which can run on affordable commodity
hardware. One of the routers built on that framework is \Ac{vpp} which
was developed by Cisco and is now open source and maintained by
\Ac{fdio}.

% little indepencent and comparable benchmarks of vpp and others

There is little independent performance analysis of \Ac{vpp} though and even less that allows for an accurate comparison between different software routers. 

% therefore this work: well documented for comparison, automated for reproducability 

% tests shall 
% - find numerical answers to the performance question
% - make vpp comparable to other software routers
% - model to describe performance behaviour
% - indentify/quantify bottlenecks

Therefore this work presents well documented performance tests and
results of \Ac{vpp} in different scenarios to enable comparability to
other software routers. The tests procedures are as far as possible
automated, to be as reproducable as possible. The resulting data is
then analyzed to learn about the behaviour of \Ac{vpp} and it's
performance bottlenecks. 

% summarize tests

Namely the this paper tests the performance scaling on multiple cores, imapct of varying cpu clocks, ip version, l2fib and l3fib sizes and tasks like vxlan encapsulation. 

% therefore the structure of this paper: 
% 1. previous performance analysis of routers in general and vpp in particular
% 2. analysis of performance critical aspects of software routing, benchmarking challenges and vpp
% 3. describe methodology and executed tests
% 4. evaluate results and create model

First, this paper will introduce previously existing methods for
router perfomance analysis and existing benchmarking results for
\Ac{vpp} in particular. Then challenges and perfomance critical
aspects of software routing and \Ac{vpp} are analyzed. In section
\ref{sec:methodology} the methodology and the executed tests are
desrcibed in detail. Finally the results are evaluated to create a
model to describe \Ac{vpp}'s performance behaviour.
