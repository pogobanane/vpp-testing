
\chapter{Introduction}

% microservices -> virtualization -> software routing
% problem: software router slower than specialized hardware

A world of microservices counting on virtualization is more and
more interested in being able to move dedicated hardware into
virtualized spaces. But especially high speed networking is
traditionally bound to specialized hardware, because many software
routers can't keep up performance-wise.

% solution: faster software routers like VPP
% vpp: cisco

There are software routers though which focus on performance and are
improved in many aspects. Especially frameworks like DPDK
\cite{inteldpdk} help in the making of fast software routers which can
run on affordable commodity hardware. One of the routers built on that
framework is \Ac{vpp} which was developed by Cisco and is now open
source and maintained by \Ac{fdio}. \Ac{vpp} can be virtualized and
has packet input nodes for NICs handled by \Ac{dpdk} (dpdk-input),
but can also use virtual links like virtual Linux interfaces (tuntap,
af-packet-input, tapcli) or memory map based interfaces (memif,
netmap, vhost-user) which allow for good interoperability with local
virtualized systems. Therefore it is promising for highly virtualized
environments.

% little indepencent and comparable benchmarks of vpp and others

% therefore this work: well documented for comparison, automated for reproducability 

% tests shall 
% - find numerical answers to the performance question
% - make vpp comparable to other software routers
% - model to describe performance behaviour
% - indentify/quantify bottlenecks

There is little independent performance analysis of \Ac{vpp} though
and none that compares it to other software routers. Therefore this
work presents well documented \cite{my:repo} performance tests and
results of \Ac{vpp} in different scenarios to enable comparability to
other software routers. The tests procedures are as far as possible
automated, to be as reproducible as possible. The resulting data is
then analyzed to learn about the behavior of \Ac{vpp} and its
performance bottlenecks.

% therefore the structure of this paper: 
% 1. previous performance analysis of routers in general and vpp in particular
% 2. analysis of performance critical aspects of software routing, benchmarking challenges and vpp
% 3. describe methodology and executed tests
% 4. evaluate results and create model

First, this paper will introduce related work about existing methods
for router performance analysis and about existing benchmarking
results for \Ac{vpp} in particular. Then challenges and performance
critical aspects of software routing and \Ac{vpp} are analyzed. In
Chapter \ref{sec:methodology} the methodology and the executed tests
are described in detail. This paper presents measured performance for
scaling on multiple cores, impact of varying CPU clocks, IP version,
layer 2 \Ac{fib} and layer 3 \Ac{fib} sizes and tasks like \Ac{vxlan}
encapsulation. Finally the results are evaluated to create a model to
describe \Ac{vpp}'s performance behavior.
