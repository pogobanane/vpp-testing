\DeclareAcroListStyle{longtabu}{table}{%
	table = longtabu,
	table-spec = @{}>{}lX@{}
}{%

\acsetup{%
	list-style=longtabu,
	extra-style=plain,	% remove dot after long in list
	only-used=false,
}

\tabulinesep=1ex

\DeclareAcronym{iso}{
	short				= {\sc{ISO}},
	long				= {International Organization for Standardization},
	list				= {\acl{iso}.},
}
\DeclareAcronym{osi}{
	short				= {\sc{OSI}},
	long				= {Open Systems Interconnection},
	list				= {\acl{osi}.},
	extra				= {%
		Reference model for layered network architectures by the \ac{osi}.
	},
}
\DeclareAcronym{pdu}{
	short				= {\sc{PDU}},
	short-plural-form	= {\sc*{PDU}s},
	long				= {protocol data unit},
	long-plural-form	= {protocol data units},
	list				= {\Acl{pdu}.},
	extra				= {%
	Refers to a message at a specific layer of the \acs{osi} model including
	all headers and trailers of the respective layer and all layers above.
	},
}
\DeclareAcronym{sdu}{
	short				= {\sc{SDU}},
	short-plural-form	= {\sc*{SDU}s},
	long				= {service data unit},
	long-plural-form	= {service data units},
	list				= {\Acl{sdu}.},
	extra				= {%
		Refers to the payload of a message at a specific layer of the \acs{osi}
		model excluding all headers and trailers of the respective layer.
	},
}
\DeclareAcronym{mac}{
	short				= {\sc{MAC}},
	long				= {medium access control},
	list				= {\Acl{mac}.},
}
\DeclareAcronym{tcp}{
	short				= {\sc{TCP}},
	long				= {transmission control protocol},
	list				= {\Acl{tcp}.},
	extra				= {%
		Stream-oriented, reliable, transport layer protocol.
	}
}
\DeclareAcronym{udp}{
	short				= {\sc{UDP}},
	long				= {User Datagram Protocol},
	list				= {\Acl{udp}.},
	extra				= {%
		Datagram-oriented, unreliable transport layer protocol.
	}
}
\DeclareAcronym{sctp}{
	short				= {\sc{SCTP}},
	long				= {Stream Control Transmission Protocol},
	list				= {\acs{sctp}.},
	extra				= {%
		Datagram-oriented, semi-reliable transport layer protocol.
	}
}
\DeclareAcronym{vpp}{
	short				= {\sc{VPP}},
	long				= {Vector Packet Processing},
	list				= {\Acl{vpp}.},
	extra				= {%
		A fast software router.
	}
}
\DeclareAcronym{fdio}{
	short				= {\sc{FD.io}},
	long				= {the Fast Data Project},
	list				= {\Acl{fdio}.}
}
\DeclareAcronym{dpdk}{
	short				= {\sc{DPDK}},
	long				= {the Data Plane Development Kit},
	list				= {\Acl{dpdk}.},
	extra				= {%
		\url{https://www.dpdk.org/}
	}
}
\DeclareAcronym{dut}{
	short				= {\sc{DUT}},
	long				= {Device under Test},
	list				= {\Acl{dut}.}
}
\DeclareAcronym{loadgen}{
	short				= {\sc{LoadGen}},
	long				= {Load Generating Device},
	list				= {\Acl{loadgen}.}
}
\DeclareAcronym{fib}{
	short				= {\sc{fib}},
	long				= {Forwarding Information Base},
	list				= {\Acl{fib}.}
}
\DeclareAcronym{cli}{
	short				= {\sc{CLI}},
	long				= {Command Line Interface},
	list				= {\Acl{cli}.}
}
\DeclareAcronym{sdn}{
	short				= {\sc{SDN}},
	long				= {Software Defined Network},
	list				= {\Acl{sdn}.}
}
\DeclareAcronym{lisp}{
	short				= {\sc{LISP}},
	long				= {Locator ID Separation Protocol},
	list				= {\Acl{lisp}.}
}
\DeclareAcronym{vpe}{
	short				= {\sc{vPE}},
	long				= {Vritual Provider Edge},
	list				= {\Acl{vpe}.},
	extra				= {%
		A Cisco \Ac{sdn} procuct group.
	}
}
\DeclareAcronym{vmdc}{
	short				= {\sc{VMDC}},
	long				= {Virtual Multiservice Data Center},
	list				= {\Acl{vmdc}.},
	extra				= {%
		Cisco VMDC 1.x-4.x: \Ac{sdn} and cloud native product groups. 
	}
}
\DeclareAcronym{ip4}{
	short				= {\sc{IPv4}},
	long				= {Internet Protocol version 4},
	list				= {\Acl{ip4}.}
}

\DeclareAcronym{ip6}{
	short				= {\sc{IPv6}},
	long				= {Internet Protocol version 6},
	list				= {\Acl{ip6}.}
}

\DeclareAcronym{perf}{
	short				= {\sc{perf}},
	long				= {Linux Performance Tools},
	list				= {\Acl{perf}.},
	extra				= {%
		\cite{perf}.
	}
}

\DeclareAcronym{vxlan}{
	short				= {\sc{VXLAN}},
	long				= {Virtual Extensible LAN},
	list				= {\Acl{vxlan}.},
	extra				= {%
		A protocol for layer 2 ethernet packet encapsulation. Allows gateways to create over 16 million ip-tunnels for layer 2 traffic to a single remote network node.  
	}
}

\DeclareAcronym{config}{
	short				= {\sc{config}},
	long				= {Configuration File},
	list				= {\Acl{config}.},
	extra				= {%
		Referres to \Ac{vpp}'s startup configuration file.  
	}
}

\DeclareAcronym{exec}{
	short				= {\sc{exec}},
	long				= {execution file},
	list				= {\Acl{exec}.},
	extra				= {%
		Referres to \Ac{vpp}'s on startup executed file.  
	}
}

\DeclareAcronym{nic}{
	short				= {\sc{NIC}},
	long				= {Network Interface Controller},
	list				= {\Acl{nic}.}
}

\DeclareAcronym{ram}{
	short				= {\sc{RAM}},
	long				= {Random Access Memory},
	list				= {\Acl{ram}.}
}

\DeclareAcronym{hqos}{
	short				= {\sc{HQoS}},
	long				= {Hirarchical Quality of Service},
	list				= {\Acl{hqos}.},
	extra				= {%
		A hirachical type of scheduling of packet sending.  
	}
}

\DeclareAcronym{rss}{
	short				= {\sc{RSS}},
	long				= {Recieve Side Scaling},
	list				= {\Acl{rss}.},
	extra				= {%
		A \Ac{dpdk} feature for multicore packet processing with hardware acceleration support. 
	}
}

\DeclareAcronym{pos}{
	short				= {\sc{pos}},
	long				= {Plain Orchestrating Service},
	list				= {\Acl{pos}.},
	extra				= {%
		An testbed orchestration tool. \cite{GallScho18} 
	}
}

\DeclareAcronym{vnf}{
	short				= {\sc{VNF}},
	long				= {Virutal Network Function},
	list				= {\Acl{vnf}.}
}

\DeclareAcronym{ndr}{
	short				= {\sc{ndr}},
	long				= {No Drop Rate},
	list				= {\Acl{ndr}.},
	extra				= {%
		The maximum throughput which doesn't result in packet drops.
	}
}